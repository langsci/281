\chapter{Introduction}
\label{chapter_intro}

Prosody has been characterised as a ``half-tamed savage" \citep[475]{Bolinger1978} being shaped by both categorical and continuous aspects. According to this view, the categorical, ``tamed" side of prosody represents those aspects that are grammaticalised and as such are part of a phonological, symbolic system. The continuous, ``untamed" side represents the ``unusually generous scope that speakers have [...] in the phonetic implementation" of prosodic categories \citep[49]{Gussenhoven2004}.\footnote{In this view, the continuous aspects of prosody can become grammaticalised in the course of language change and thus develop to be used in a categorical fashion.}

The objective of many approaches to prosody has been to ``draw a sharp dividing line between the tamed half and the untamed half" \citep[49]{Gussenhoven2004}. Interestingly, this aim resonates with a more general, long-standing debate in linguistics revolving around the question how phonology with its categorical representations and phonetics with its continuous signals are related. The prevalent view has long been that  phonological knowledge, the mental representations of speech sounds, is best conceptualised as symbols and discrete rules or constraints that operate on these symbols in an abstract system. The result of the discrete computations has to be translated into a continuous, phonetic signal \citep{Ladd2006}, an acoustic output produced via articulatory movements. The relation between phonology and phonetics is thus characterised by a translation of categorical to continuous, involving two fundamentally different ``formal languages" \citep[906]{GafosBenus2006}. Fruitful proposals to solve problems arising from the disparity of these representations are rooted in the framework of nonlinear dynamical systems. The framework has gained increasing attention in modelling phenomena in cognition \citep[among others][]{Kelso1995, vanGelderPort1995, Gafos2006, GafosBenus2006, Port2002, Spivey2007, ThelenSmith1994, Tulleretal1994} because it can provide one formal language to capture categorical and continuous aspects of cognition at the same time.

The present work aims to shed light on the relation between the categoriality and the continuity of prosodic prominence. Crucially, it argues that it is often difficult to draw a sharp dividing line between the tamed and the untamed sides of prosody. It demonstrates that what has been termed phonetic or ``untamed" seems to form synergies with what has been described as phonological or ``tamed", and both work jointly towards the same communicative goals. In particular, the present work investigates recordings of 27 native speakers of German marking focus types in an interactive task. The analysis thereby pursues \emph{integration} in a two-fold manner. On the one hand, the integration of categorical and continuous aspects of pitch accents is examined, revealing that the probabilistic mapping of focus types to pitch accent categories is mimicked by the continuous parameters of the pitch accents. For instance, a higher probability of rising accents is accompanied by larger pitch excursions of these rising accents. On the other hand, the present work integrates multiple dimensions of prosodic focus marking by combining the tonal analyses with investigations of articulatory movements of the lips and the tongue, showing that speakers make use of a rich set of parameters.

These results are incorporated into a dynamical approach that models the discreteness of phonological categories and the continuous nature of phonetic substance as well as the multi-dimensionality of prosodic patterns. The present work thereby emphasises the synergies of categorical and continuous aspects of prosody and questions the need to be able to separate the ``tamed" and the ``untamed" sides in a theoretical approach. The aim is to contribute to a larger understanding of how ``a symbiosis of the symbolic and subsymbolic paradigms” \citep[19]{Smolensky1988} can be developed. On the one hand, this symbiosis is desirable in order to bridge gaps between the disciplines of phonetics and phonology. On the other hand, the symbiosis is needed to form analytical synergies that can cope with a growing body of findings  demonstrating that the sound patterns of language are characterised by a wide array of variability, gradient phenomena and systematic fine-grained details.

The book is structured as follows:


\textsc{Chapter} \ref{chapter_pandp} sheds light on the relation of phonetics and phonology, or how the relation of the two has been conceptualised in theoretical frameworks. After very briefly tracing the history of some of the most important ideas that led to today's understanding of phonetics and phonology, the chapter turns to problems that arise from a strict separation of phonetics and phonology, or a purely symbolic phonology. In this context, phenomena like (in)variance of sound categories in the world's languages, assimilation, vowel harmony, and incomplete neutralisation are discussed. Special attention is paid to solutions that are provided by models to cope with these phenomena, such as optional rules, the introduction of stochasticity and scalar values in Optimality Theory, gestural overlap in Articulatory phonology, phonetic implementation rules, and concepts of exemplar\largerpage{} theory. 

\textsc{Chapter} \ref{chapter_ds} introduces the framework of dynamical systems. This chapter has two parts. In the first part, the basic concepts of dynamical systems and attractors are presented. The most important features of dynamical systems are illustrated using the logistic map, and differential equations and multi-stability are introduced. In the second part of the chapter, applications of dynamical models in phonetics, phonology and beyond are investigated in more detail: the \citet{HakenKelsoBunz1985} model of inter-limb coordination patterns, the harmonic oscillator of Articulatory phonology, the coupled oscillator model for the coordination of speech gestures, the categorical perception model of \cite{Tulleretal1994}, and two models by \citet{GafosBenus2006} for incomplete neutralisation and Hungarian vowel harmony. This second part takes up the problems of the relation of phonetics and phonology outlined in Chapter \ref{chapter_pandp}. It describes how models based on dynamical systems can help to learn more about the relation of phonetics and phonology, and how attractors in dynamical systems relate to the conception of linguistic categories in a traditional sense. Throughout the chapter, the models are illustrated using MATLAB code that accompanies this book and is available for download.

\textsc{Chapter} \ref{chapter_prosody} deals with the topic of prosody and prosodic prominence. The chapter sketches some of the concepts that are fundamental to the study presented later in the book, such as pitch accents, prosodic structure and prosodic strengthening. Special attention is paid to what has been described as categorical and what has been described as continuous in prosody research. This chapter introduces the prosody of focus marking -- a field of research that the present work attempts to contribute to. The chapter takes three perspectives on focus marking by incrementally adding bits of evidence. It reviews what is known about the tonal and articulatory patterns of prosodic focus marking and refines the view on categorical and continuous phenomena in these patterns. In doing so, it narrows down the subject of the empirical and modelling part of the book. 

\textsc{Chapter} \ref{chapter_data} introduces the objectives of the empirical part of this work and describes the experimental methods used to collect the data that are analysed in the next two chapters. The corpus of collected data described here comprises productions of 27 native speakers of German marking different focus structures (background, broad focus, narrow focus, and contrastive focus) by means of prosody in a controlled experimental environment.

\textsc{Chapter} \ref{chapter_onglide_modelling} presents the results of the F0 measures of a subset of productions of the corpus. This subset contains all utterances in which the nuclear accent is placed on the target word, i.e. the indirect object. The analysis demonstrates that speakers use categorical and continuous modulations of F0 to mark focus types and that the two types of modulation form a symbiosis. A first dynamical model for pitch accents is sketched in the second part of this chapter. This model represents an account that reconciles the categoriality and the continuity of pitch accents found in the data. As a result, it provides a first step towards an approach that dispenses with a strict division of abstract representations of pitch accents on the one hand, and their phonetic implementation in terms of F0 on the other.

\textsc{Chapter} \ref{chapter_multi_prosody} extends the analysis of F0 patterns including the utterances without nuclear pitch accent on the target word and adds measures of articulatory movements of the lips and the tongue. The chapter  sketches a second model that enriches the account of Chapter \ref{chapter_onglide_modelling} in two ways: First, the transition from unaccented to accented is conceptualised as a bifurcation, a qualitative change, in the dynamical system. Second, the attractor landscapes are understood as multi-dimensional constructs in which many dimensions -- both laryngeal and supra-laryngeal -- contribute to a complex, flexible bundle of prosodic prominence.

\textsc{Chapter} \ref{chapter_discussion} completes the book with a general discussion.


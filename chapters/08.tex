\chapter{General discussion}
\label{chapter_discussion}

The present work provided a dynamical systems approach to contribute to an integration of categorical and continuous aspects of prosody. In a multi-dimensional account that incorporates various facets of prosodic prominence, this approach fuses intonation and articulatory modulations of prosody within a joint system. The model was developed on the basis of a large EMA corpus of recordings of 27 speakers of German allowing for an extensive analysis of the tonal and articulatory patterns of marking different focus structures. The results showed that prosodic prominence entails a symbiosis of categorical and continuous, as well as tonal and articulatory adjustments and that speakers use the bundle of cues to prosodic prominence flexibly, yet systematically at the same time. The present work thereby participates in a long-standing debate that revolves around categorical and continuous phenomena in speech, and the tension between symbolic and continuous descriptions.

\section{Summary of the results and modelling approach}

The main findings of the study with regard to F0 can be summarised as follows:

\begin{enumerate}

\item The data of prosodic focus marking reveal categorical and continuous modulations at the same time.

\item The first, most obvious – and of course unsurprising – categorical modulation is the placement of a nuclear pitch accent on the target word from background to broad focus.

\item In addition, speakers use roughly equal numbers of falling / early and rising / mid to late pitch accents in broad focus but increase the number of rising / mid to late accents in narrow focus. From narrow to contrastive focus, the number of rising / mid to late accents is increased even further.

\item In addition, the magnitude of the rises, assessed here as the quantity in terms of tonal onglide, is increased and the alignment of the peak is delayed from broad to narrow focus and from narrow focus to contrastive focus.

\end{enumerate}

These results are in line with the observation of \citet{Bolinger1961} or \citet{Ladd2014} that categorical and continuous modulations in prosody are often hard to disentangle. The data of the present study show that the two types of modulation are used in symbiosis: A higher frequency of rises goes hand in hand with an increase in the magnitude of the rises. Likewise, a higher frequency of mid to late accents goes hand in hand with a later peak alignment of these accents. The analysis is congruent with the general notion of prosody as a ``half-tamed savage” \citep{Bolinger1978, Gussenhoven2004}. It is, however, questionable that it is possible to draw a ``sharp dividing line” \citep[49]{Gussenhoven2004} between the tamed and the untamed half, as envisioned by many phonological models of prosody. Rather, the symbiosis of the categorical and continuous aspects underline Bolinger's \citeyearpar[475]{Bolinger1978} claim that ``to understand the tamed or linguistically harnessed half [...] one has to make friends with the wild half".

The present data reveal that a great deal of fuzziness is involved in the prosodic modulations used by speakers to mark focus. There is no one-to-one mapping between focus types and pitch accent types. Rather, overlapping distributions are found, both in the categorical domain as well as in the continuous domain. For example, broad focus may be expressed by using a falling accent or a ``mildly” rising accent. However, narrow focus also exhibits many rather shallow rises and even some falling accents. 

This might be due to the fact that meaning differences expressed through prosody are often not as clear-cut as differences in lexical meanings in languages like German and English. However, as the discussions in Chapter \ref{chapter_pandp} and Chapter \ref{chapter_ds} showed, even when it comes to speech sound phenomena involved in the differentiation of lexical meanings (often termed the ``segmental” domain), a lot of fuzziness and variation is found. Purely symbolic approaches often have difficulties in dealing with this fuzziness and variation. 

The integration of categorical and continuous, as well as the fuzziness or probabilistic nature can well be captured in the dynamical modelling approach outlined in this book. It resonates with a view ``in which the human mind/brain typically construes the world via partially overlapping fuzzy gray areas that are drawn out over time” \citep[3]{Spivey2007}, a perspective referred to as ``the continuity of mind”. 

The models sketched here use the concept of the attractor to induce stability that is comparable to the notion of prosodic categories. The attractors vary in stability when the control parameter is scaled, providing a mechanism to capture the overlapping of the tonal onglide distributions. Since the attractors are stable states on continuous dimensions, the implementation of the categories follows directly from the location of the attractors themselves. In other words, there is no separation between an abstract category and a concrete implementation. This allows for fuzziness in the physical output of the system. As the control parameter is changed and the attractor landscape tilts towards the ``rising” side, the rising attractor stabilises and also shifts subtly to the right. Hence, comparing the rising (positive) parts of outcome distributions of the different control parameter values (e.g. by simulation), there is of course a lot of overlap but also a general trend towards more extreme values in addition to more rises in general – exactly as in the empirical data. 

The analysis of speaker groups reveals that speakers use this system in different but comparable ways. For both speaker groups, the differentiation of broad from narrow and of narrow from contrastive focus can be accounted for by a stabilising of the rising attractor, i.e. an increase in the control parameter. However, the speaker groups differ with respect to the range of control parameter values they use. Group 1 starts lower than group 2. As a consequence, the speakers of group 2 have almost exclusively rising accents in their repertoire, the speakers of group 1 have considerable numbers of rising \emph{and} falling accents. The speaker groups mirror the main speaker strategies found in \citet{Griceetal2017}. In this study, it was demonstrated that all speakers manipulate the F0 parameters under scrutiny in the same direction. In doing so, some speakers crossed category boundaries, e.g. negative tonal onglides in broad focus and positive onglides in narrow focus, while other speakers modulated the parameters within the boundaries of the same pitch accent category. The former group of speakers corresponds to group 1 in the current work, the latter corresponds to group 2. The dynamical model provides a formal implementation of the generalisation that the parameters are manipulated \emph{in the same direction}. It even goes a step further in employing the control parameter as a device to scale the prosodic prominence. The common strategy of both speaker groups is that they increase the control parameter from broad to narrow focus and from narrow to contrastive focus, the difference lies in the range of values that are used for the control parameter.

The qualitative change in the data from background to broad -- a unimodal distribution of onglide values changes into a bimodal distribution -- is accounted for by a bifurcation in the model of Chapter \ref{chapter_multi_prosody}. In a classical phonological sense, this change in the system corresponds to the placement of an accent. This part of the modelling reveals one of the major strengths of dynamical systems. In addition to the more gradual modifications (between broad, narrow and contrastive focus), the system is able to exhibit a ``dramatic" change (between background and broad focus). Crucially, the cause of the qualitative change is of the same kind as the cause of the gradual changes, both kinds of change are effected by the scaling of the same control parameter.

In addition to an extensive analysis of the tonal pattern in the data, the book provided results in the domain of supra-laryngeal articulation. With 27 speakers and 2088 productions of the target sentence, the current data collection represents a large EMA corpus contributing to our understanding of the role of articulatory modulations in prosody. The results of the supra-laryngeal articulatory parameters can be summarised as follows:

\begin{enumerate}

\item The lips are opened wider from background to broad focus (unaccented $\rightarrow$ accented) and from broad to contrastive focus (within accentuation) with an intermediate step for narrow focus, i.e. within the group of accented target words.

\item The tongue body is lowered from background to broad focus (unaccented $\rightarrow$ accented) and from broad to contrastive focus (within accentuation) with an intermediate step for narrow focus (within the group of accented target words).

\item In the vowel /o/, the tongue body is retracted  from background to broad focus (unaccented $\rightarrow$ accented) and from broad to contrastive focus (within accentuation) with an intermediate step for narrow focus (within the group of accented target words), although these results are not as clear as those for the vertical tongue body position.

\end{enumerate}

Again, the distributions reveal a great deal of overlap or fuzziness. This is particularly evident when it comes to the differentiation of narrow focus from the two ``neighbouring" focus types broad and contrastive focus. Narrow focus overlaps with both of them, the statistical results are not always clear. However, a general trend for a continuous increase in prosodic prominence beyond accentuation expressed by a larger opening of the vocal tract and more peripheral vowel articulation, can be attested. The articulatory results were used to extend the model outlined for tonal onglide to include more dimensions. In this model, the prosodic dimensions contribute differently to the shared attractor landscape. The complexity of the articulatory dimensions is lower compared to the tonal onglide dimensions as they do not exhibit bifurcation or the presence of two attractors. An important point conveyed by the modelling approach is that change on all dimensions is induced by the scaling of the control parameter of the system. 

\section{Limitations and future directions}

Like many models, the current approach has some limitations. First, as already discussed in Chapter \ref{chapter_onglide_modelling}, the results of the speaker groups do not fit perfectly the predictions of the model. The differences between the rising means of group 1 compared to group 2 are predicted to be larger than they are in the real data. It is plausible to assume that the tonal onglide, albeit being a very important parameter, does not capture all details in the tonal domain. Speakers certainly mark prominences and differentiate pragmatic meanings on many dimensions \citep{BaumannWinter2018}.

Another regard in which the model is simplifying is the domain of tonal movement. The analyses look at a stretch of the F0 contours that has been shown to be a critical part \citep{RitterGrice2015, BaumannRöhr2015}. However, when it comes to the flat contours in background, it is not entirely clear how the arbitrary window translates into a production unit. From an AM perspective, where contours are underspecified, it seems unusual to specify tonal onglide values for this low stretch of F0 as it only represents an interpolation between two tonal targets. It is not the intention of the model to state that speakers choose certain F0 values in the arbitrarily chosen time window. However, what is going on in this time window is representative for the intonation contour as such. It is thus better to take the characterisation that the measure is able to provide as a representation of a larger unit (for example, in the case of background this could be something like ``everything after the direct object should sound flat or low"). Likewise, with regard to the nuclear pitch accents, it should be noted that -- despite choosing the tonal onglide -- the study intends to leave open as to whether portions of F0 after the tonal target on the accented syllable (``offglide" portions) might play an important role like the data of \citet{KüglerGollrad2015} suggest.

Using a single control parameter for multiple dimensions, in the way the model does in the current version, leads to the simplifying assumption that the relation of scaling on all dimensions is proportional for all speakers. It is, however, not implausible to assume that speakers might weight one dimensions more in relation to another. For example, a speaker could concentrate on sonority expansion and show large differences between the focus types with regard to lip aperture, but use only slight modulations in terms of F0. Another speaker might exhibit large changes in F0 but only subtle changes of lip aperture. These fine-grained differences are not captured by the model in the current form. One the one hand, the analysis needs to be extended on speaker-specific strategies in all prosodic domains, including articulation. On the other hand, the model might need additional weights or control parameters. In extending the model, the interconnection of dimensions should be maintained. How this is to be achieved has to be left for future research.

Another task of future research will be to consolidate the general idea that a dynamical systems approach is the right avenue to pursue in modelling prosody and that a plausible type of dynamical model is assumed. This work already showed that scaling the control parameter of the system in the current form may lead to a bifurcation. From a dynamical modelling point of view, this property is important, as \citet[1538]{Kelso2013} puts it: ``you only know for certain you have identified a control parameter if, when varied, it causes the system’s behaviour to change qualitatively or discontinuously". One important problem of the current approach is, however, that it assumed a dynamical formulation by investigating only the equilibrium positions of the putative system, i.e. the attractors where the system comes to rest. The trajectories, i.e. the paths towards the attractors, were not included in the description and in fact they may be impossible to observe. It has also to be borne in mind that the trajectories in the proposed model are shaped by noise to a large degree. Further work on this topic will have to deal with this limitation. In addition, it might be promising to test other important traits of dynamical systems such as sensitivity to initial conditions or hysteresis, although here as well the implications of noise have to be taken into account. 

Finally, the work presented here was based on production data alone. In future works on the topic, the relevance of the prosodic modulations for the decoding of pragmatic meanings should be investigated in more detail. Data of the kind presented here can be used in a perception task. It may also be feasible to model the patterns of perception of the prosodic data in a dynamical framework. In particular, the perception of speakers employing different strategies (like group 1 vs. group 2 in the current data set) may be an interesting research topic. From the modelling perspective in this work, it could be assumed that listeners adjust their attractor landscapes or scalings of attractor landscapes according to a given speaker they listen to. In addition, interlocutors may also be able to align their uses of the system in interaction as an effect of accommodation or entrainment.

\section{Conclusion}

The current book provided an analysis of tonal and articulatory strategies of prosodic focus marking in German. It investigated the implications of the obtained results for the relation between the categorical and the continuous aspects of prosodic prominence -- thereby contributing to a long-standing debate on the relation between phonology and phonetics. While categorical and continuous modulations seem to go hand in hand and often form a kind of symbiosis, it is desirable to reconcile the two in a joint theoretical treatment. The work presented here followed work on models of cognition in the framework of dynamical systems and sketched out an attractor modelling approach for the patterns of prosodic prominence in the data. This modelling account is able to unify the categorical as well as the continuous aspects of prosodic prominence on one level, and therefore provides a promising tool to integrate the phonology and phonetics of prosody. In addition, it offers a first outline of an integration of different tiers or dimensions that contribute to prosodic prominence, tonal and articulatory, that have often been treated separately. Future research will investigate the patterns of prosodic prominence in both production and perception in more detail and consolidate the idea of prosody as a multi-dimensional dynamical system.

